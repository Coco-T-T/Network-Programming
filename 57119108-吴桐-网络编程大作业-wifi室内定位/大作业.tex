\documentclass{article}

\usepackage{geometry}

\geometry{a4paper}

%\usepackage[UTF8, heading = false, scheme = plain]{ctex}%格式

\usepackage{ctex}

%\usepackage{authblk} %添加机构 安装preprint后可用

\usepackage{graphicx} %添加图片

\usepackage{amsthm}

\usepackage{amsmath}

\renewcommand{\vec}[1]{\boldsymbol{#1}} % 生产粗体向量,而不是带箭头的向量

\usepackage{amssymb}

\usepackage{booktabs} % excel导出的大表格

\usepackage{hyperref}

%\newtheorem{definition}{Definition} %英文

%\newtheorem{theorem}{Theorem}

\newtheorem{definition}{定义} %中文

\newtheorem{lemma}{引理}

\newtheorem{theorem}{定理}

%\newenvironment{proof}{{\noindent\it 证明}\quad}{\hfill □ \square□\par}

\DeclareMathOperator{\Ima}{Im}%定义新符号

\DeclareMathOperator{\Rank}{rank}%定义求秩算子

\title{基于WiFi信号强度的室内定位系统}

\author{吴桐\\57119108}

%\affil{中国科学院} 
%需要把上面的\usepackage{authblk}取消注释

%date{}

\begin{document}

\maketitle

%\tableofcontents

%\newpage

%\listoffigures

%\newpage

\section{系统设计思路}

本次课程设计的目的,是在一个已知所有AP信息(ID、位置等)的室内环境,根据设备所探测到的WiFi信号强度,实现室内定位(导航)。

1、探测WiFi信号强度

pywifi是python中一个跨平台的用于操作无线接口的模块,支持Windows和Linux系统。
利用pywifi模块可以实现WiFi信号的探测,获取AP的SSID、MAC地址以及信号强度。

\begin{verbatim}
import pywifi
from pywifi import const, Profile

# 获取网卡接口
wifi = pywifi.PyWiFi() 
# 得到第一个无线网卡
ifaces = wifi.interfaces()[0]
# 扫描wifi(AP)
ifaces.scan()
# 查看上面的wifi扫描结果(返回值为列表)
result = ifaces.scan_results()
\end{verbatim}

2、根据WiFi信号强度估算位置

在获取到WiFi信号及其强度后,我们可以利用信号衰减公式,通过信号强度估算接收设备与AP之间的距离。
由于我们已知各个AP的位置,在估算出接收设备与AP之间的距离后,就可以使用“三点定位算法”,计算出接收机的位置。

“三点定位算法”需要已知三个确定的点,我们选择接收到的WiFi信号中最强的三个信号,作为计算依据。

3、在地图中绘制自己的位置

我们需要对一个特定的室内环境进行考察,绘制出地图(示意图)。这里我们使用python进行绘图,在二维空间内标识出各个AP的位置以及接收设备所在的位置。
   
\section{探测WiFi信号强度}

如上一节所述,我们利用pywifi模块获取室内WiFi信号列表。注意,因为扫描需要时间,所以在调用$ifaces.scan()$后需要等待一段时间(一般是5-8s)再获取WiFi列表。
在扫描结果中,我们会发现重复的WiFi信号,因此这里我们使用$set$去除重复的AP。

由于我们只对梅园食堂一楼的室内环境进行测试,在程序中预设了梅园食堂一楼的AP信息,因此我们可以直接过滤掉扫描到的未知AP,这并不会影响最终的定位结果。

\begin{verbatim}
def getWifiList():
    MACset.clear() # 初始化
    APlist.clear()
    ifaces.scan()
    time.sleep(5)
    results = ifaces.scan_results()  #扫描所有wifi
    for result in results:
        # MAC地址需要在已知的字典中
        if result.ssid == 'SEU-WLAN' and result.bssid in APdict:
            if result.bssid not in MACset:
                MACset.add(result.bssid)
                obj = AP(result.ssid, result.bssid, result.signal)
                APlist.append(obj) 
\end{verbatim}

我们对得到的去重、去未知的AP列表进行排序,取信号强度最大的三个AP,用于之后的定位算法。

\begin{verbatim}
def findMaxWifi(): 
    APlist_sort = sorted(APlist, key=lambda x:x.signal, reverse=True)
    return [APlist_sort[0],APlist_sort[1],APlist_sort[2]]
\end{verbatim}

\section{三点定位算法}

三点定位法需要已知圆的圆心和半径,求三圆的交点(图1)。

\begin{figure}[ht] %htbp

    \centering

    \includegraphics[scale=0.1]{图11.png}

    \caption{三点定位}

    \label{fig:j}

\end{figure}

利用毕达哥拉斯定理我们可以快速求得交点O的坐标。然而,现实情况并不总是如人意。
如图2所示,三个圆相交可能得到一个公共区域而非公共点。甚至,某些情况下三个圆完全不相交。

\begin{figure}[ht] %htbp

    \centering

    \includegraphics[scale=0.1]{图12.png}

    \caption{三圆相交得到公共区域}

    \label{fig:k}

\end{figure}

由于各种原因,想要准确地知道接收设备的当前位置是相当困难的,但是我们可以通过运算得到一个近似解。
我们从两个圆入手,先找到两两圆之间的中心点,再求三个中心点的平均得到三圆的中心点。

除去圆中圆(现实情况基本不存在),两圆关系可以分为两种:相交和不相交。

考虑两个圆相交的情况(图3),此时两圆交点为A和B,我们想要求得AB的中点C。
根据勾股定理可知:

\begin{equation}\label{q}
	\left\{
        \begin{aligned}
            &PQ=PC+QC  \\
            &{QA}^{2}={QC}^{2}+{AC}^{2} \\
            &{PA}^{2}={PC}^{2}+{AC}^{2}
        \end{aligned}
    \right.
\end{equation}

解得:

\begin{equation}\label{w}
	PC = \frac{PQ}{2} + \frac{{PA}^{2}-{QA}^{2}}{2PQ}
\end{equation}

根据比例关系求得坐标:

\begin{equation}\label{e}
    \left\{
        \begin{aligned}
            x&={x}_{P} + \frac{({x}_{Q}-{x}_{P})PC}{PQ}  \\
            y&={y}_{P} + \frac{({y}_{Q}-{y}_{P})PC}{PQ}
        \end{aligned}
    \right.
\end{equation}

\begin{figure}[ht] %htbp

    \centering

    \includegraphics[scale=0.1]{图14.png}

    \caption{两圆相交}

    \label{fig:m}

\end{figure}

当两圆不相交时,简单地,我们根据半径比例计算坐标:

\begin{equation}\label{r}
    \left\{
        \begin{aligned}
            &sum={r}_{P} + {r}_{Q} \\
            &x={x}_{P} + \frac{({x}_{Q}-{x}_{P}){r}_{P}}{sum}  \\
            &y={y}_{P} + \frac{({y}_{Q}-{y}_{P}){r}_{P}}{sum}
        \end{aligned}
    \right.	
\end{equation}

代码详见loc.py。

然而,在实验中我们发现,一个AP可能存在多根发射天线,接收到的不同MAC地址的信号实际上来自同一AP。我们选择信号强度最大的三个信号作为计算依据,若三个选中的信号中出现“来自的AP相同”的情况时,我们就无法使用“三点定位算法”。
此时我们采用的定位算法:若最强的三个信号来自两个不同的AP,则将两个AP位置连线的中点作为估计位置;若最强的三个信号来自同一个的AP,则将AP的位置作为估计位置。

还有一个问题:如何根据信号强度估算接收设备与AP之间的距离。资料显示自由空间中电磁波传播的损耗公式为:

\begin{equation}\label{t}
	Los = 32.44 + 20lg(D)(Km) + 20lg(F)(MHz)
\end{equation}

其中,Los是传播损耗;D是距离,单位是Km;F是工作频率,单位是MHz。理论上,若已知传播损耗(相对于10cm以内AP信号强度)和工作频率,我们可以计算出接收设备与AP之间的距离。

然而,在实际测试中发现,简单地套用公式的效果非常不理想,可能是因为室内空间存在大量遮挡物。
因此,在最终的程序里,我对以上公式做了一些参数调整,使结果更加符合实际测试环境。

\section{绘制地图}

通过实地考察,我们确定了梅园食堂一楼各个AP的位置和相应的MAC地址,并将其映射关系预设到程序中。
为了方便描述与测试,我们给各个AP编号,如图4所示。其中,(0,0)点对应梅园食堂一楼的东北角,地图符合上北下南的规则。

\begin{figure}[ht] %htbp
    
    \centering
    
    \includegraphics[scale=0.5]{图9.png}
        
    \caption{梅园食堂一楼AP分布示意图}
        
    \label{fig:i}
        
\end{figure}

\section{测试结果}

我在食堂的多个地点进行了测试(图6-11)。
在测试过程中,我发现AP的信号质量对定位结果的准确性有很大影响。
与其余AP相比,7号AP的信号强度较弱,当接收设备处在7号和15号AP之间时,检测到的WiFi信号中最强的三个信号,来自于3号和15号AP,最终的定位结果准确度就会下降。

\begin{figure}[ht] %htbp

    \centering

    \includegraphics[scale=0.4]{图2.png}
    
    \caption{控制台界面}
    
    \label{fig:b}
    
\end{figure}

\begin{figure}[ht] %htbp

    \centering
    
    \includegraphics[scale=0.5]{图3.png}
    
    \caption{测试结果1}
    
    \label{fig:c}
    
\end{figure}
    
\begin{figure}[ht] %htbp
    
    \centering
    
    \includegraphics[scale=0.5]{图4.png}
        
    \caption{测试结果2}
        
    \label{fig:d}
        
\end{figure}

\begin{figure}[ht] %htbp
    
    \centering
    
    \includegraphics[scale=0.45]{图5.png}
        
    \caption{测试结果3}
        
    \label{fig:e}
        
\end{figure}

\begin{figure}[ht] %htbp
    
    \centering
    
    \includegraphics[scale=0.5]{图6.png}
        
    \caption{测试结果4}
        
    \label{fig:f}
        
\end{figure}

\begin{figure}[ht] %htbp
    
    \centering
    
    \includegraphics[scale=0.48]{图7.png}
        
    \caption{测试结果5}
        
    \label{fig:g}
        
\end{figure}

\begin{figure}[ht] %htbp
    
    \centering
    
    \includegraphics[scale=0.48]{图8.png}
        
    \caption{测试结果6}
        
    \label{fig:h}
        
\end{figure}
    
\section{总结}

通过本次课程设计,锻炼了我的综合能力。在系统设计过程中,我尽可能考虑了代码的鲁棒性,为各种特殊情况都设置了解决方案;同时,我也充分考虑了程序的用户友好性,设计了控制台命令引导;
在算法设计过程中,我通过查阅资料,最终确定了简单而高效的“三点定位算法”;在实验过程中,我尽量还原室内场景的布置,反复测试每个AP的MAC地址,
感谢老师和同学们在本次课程设计中提供的帮助,最终呈现出来的效果还是很不错的。

\end{document}
